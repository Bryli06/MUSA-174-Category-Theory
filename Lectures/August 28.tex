\section{Lecture 8/28}
\subsection{Categories}
First consider all the algebraic structures we know:

\begin{center}
\begin{tabular}{c|c}
     Structure & Action \\\hline
     Set & Functions \\\hline
     Group & Group Homomorphism \\\hline
     Ring & Ring Homomorphism \\\hline
     Field & Field homomorphism \\\hline
     Vector Space & linear transformation \\
\end{tabular}
\end{center}

Now we wish to abstract all these properties through category theory. Namely, an object and a morphism. 

\begin{definition}
    A category $\mathcal{C}$ consists of a:
    \begin{itemize}
        \item a collection of objects (c, d, x, y, z, \dots)
        \item a collection of morphisms $(f: x\to y, g: y \to z) $
    \end{itemize}
    that has the following Data:
    \begin{enumerate} [label = D\arabic*\textrangle]
        \item Every morphism $f: x\to y$ has domain $x$ and codomain $y$ where $x,y$ are objects in the collection 
        \item For $f: x\to y, y: y\to z, \exists g \circ f: x \to z$ in the collection of morphisms 
        \item There is a morphism $e_x: x\to x$ s.t. $\forall f: x \to y, e_y f = f e_x = f.$ This morphism is later defined to be the identity.
    \end{enumerate}
     
    Furthermore, $\mathcal{C}$ satisfies the following axioms:
    \begin{enumerate} [label = A\arabic*\textrangle]
        \item The identity $e_x$ satisfies $e_y f = f e_x = f$
        \item (Associativity) If $x \xrightarrow{f} y \xrightarrow{g} z \xrightarrow{h}w,$ then $(hg)f = h(gf).$
    \end{enumerate}
\end{definition}

\begin{remark}
    Notice that we do not define the identity $e_x$ as $\text{id}(x) = x$ as $x$ is not necessarily a set. 
\end{remark}

\begin{remark}
    We explicitly use the term 'collection" instead of the term 'set' as to avoid the details of sets like Russell's paradox. What a 'collection' exactly is will be covered later.
\end{remark}
\begin{idea}
    Definitions in Category Theory are cofused on how functions change, not how the underlying elements change by the function.
\end{idea}

\subsection{Poset Categories}
\begin{definition}
A \textbf{particularly ordered set (poset)} $(\mathcal{P}, \le)$ is a set $\mathcal{P}$ with ordering $\le$ such that:
\begin{enumerate} [label=\arabic*\textrangle]
    \item $a\le a$
    \item $a\le b, b\le c \implies a \le c$
    \item $a\le b, b\le a, \implies a = b$
\end{enumerate}
\begin{example}
The following are posets:
\begin{itemize}
    \item $(\mathbb{N}, |)$ $  a|b, b|c \implies a|c$
    \item $(\mathbb{R}, \le)$
    \item $(S, \in),$ for some set $S$ as $A\in B, B\in C \implies A\in C$
\end{itemize}
\end{example}
\end{definition}

Now, we claim that posets form a category
\begin{definition} [Poset Categories]c
A \textbf{poset category} has object which are elements of $\mathcal{P}$ a \textit{unique} morphism $f:x\to y$ iff $x\le y \in \mathcal{P}.$
    
\end{definition}

\begin{proof}
$D1$ to $D3$ and $A1$ follow naturally where the identity $e: x\to x$ is defined by $a \le a.$ Similarly, $A2$ follows from the property of transitivity by posets.     
\end{proof}

\begin{remark}
    Note the importance of defining the morphism as unique as transitivity guarantees a unique morphism $\le$ from $x\to z$
    % https://q.uiver.app/#q=WzAsMyxbMCwwLCJ4Il0sWzEsMCwieSJdLFsyLDAsInoiXSxbMCwxLCJmIl0sWzEsMiwiZyJdLFswLDIsImgiLDAseyJjdXJ2ZSI6LTN9XSxbMCwyLCJoJyIsMix7ImN1cnZlIjozfV1d
\[\begin{tikzcd}
	x & y & z
	\arrow["f", from=1-1, to=1-2]
	\arrow["h", curve={height=-18pt}, from=1-1, to=1-3]
	\arrow["{h'}"', curve={height=18pt}, from=1-1, to=1-3]
	\arrow["g", from=1-2, to=1-3]
\end{tikzcd}\]
\end{remark}