\section{Lecture 8/30}
\subsection{Examples of Categories}
We spent this class reviewing the definition of a category and going through examples.
First consider the following diagrams. Note that they are categories with $\rightarrow$ being the morphism and $\bullet$ being objects.

\begin{center}

\begin{tabular}{cccc}
% https://q.uiver.app/#q=WzAsMixbMCwwLCJcXGJ1bGxldCJdLFsxLDAsIlxcYnVsbGV0Il0sWzAsMV1d
\begin{tikzcd}
	\bullet & \bullet
	\arrow[from=1-1, to=1-2]
\end{tikzcd} &
% https://q.uiver.app/#q=WzAsMyxbMCwxLCJcXGJ1bGxldCJdLFsxLDEsIlxcYnVsbGV0Il0sWzEsMF0sWzAsMSwiIiwwLHsiY3VydmUiOjF9XSxbMSwwLCIiLDAseyJjdXJ2ZSI6MX1dXQ==
\begin{tikzcd}
	& {} \\
	\bullet & \bullet
	\arrow[curve={height=6pt}, from=2-1, to=2-2]
	\arrow[curve={height=6pt}, from=2-2, to=2-1]
\end{tikzcd} &
% https://q.uiver.app/#q=WzAsNCxbMCwxLCJcXGJ1bGxldCJdLFsxLDEsIlxcYnVsbGV0Il0sWzEsMF0sWzAsMCwiXFxidWxsZXQiXSxbMywwXSxbMSwwXV0=
\begin{tikzcd}
	\bullet & {} \\
	\bullet & \bullet
	\arrow[from=1-1, to=2-1]
	\arrow[from=2-2, to=2-1]
\end{tikzcd} & 
% https://q.uiver.app/#q=WzAsMyxbMCwxLCJcXGJ1bGxldCJdLFsxLDEsIlxcYnVsbGV0Il0sWzEsMF1d
\begin{tikzcd}
	& {} \\
	\bullet & \bullet
\end{tikzcd}
\end{tabular}
\end{center}

\begin{remark}
    The identity morphism (an arrow back to itself) has been omitted in all of these diagrams as it is implied. 
\end{remark}

Notice that in the last example, there are no morphisms (other than the identity). This is a special case known as a discrete category.

\begin{definition}
    A category is discrete if all morphisms are identity.
\end{definition}

Note that given any set $\mathcal{S},$ we can generate a discrete category $S_d$ where 
\begin{itemize}
    \item The objects are elements of $\mathcal{S}$
    \item The morphisms are the set of identities
\end{itemize}

Now recall the table of structures \ref{table:structure-table}. We will construct different categories using groups, vector spaces, etc. 

\begin{definition} [Deloopings]
    Let $G$ be a group. The delooping of $G, \mathcal{B}G,$ is defined as 
    \begin{itemize}
        \item $\obj (\mathcal{B}G) = \star,$ where $\star$ is any single element, potentially unrelated to the group. 
        \item $\mor (\mathcal{B}G) = \{ g \in G\}$
    \end{itemize}
\end{definition}

\begin{example} [Examples of Deloopings] 
Think of deloopings as group actions, such as the permutations of a cube. 
\begin{center}
\begin{tabular}{cc}
% https://q.uiver.app/#q=WzAsMSxbMCwwLCJcXHN0YXIiXSxbMCwwLCJmZyIsMCx7Im9mZnNldCI6LTR9XSxbMCwwLCJnIiwwLHsib2Zmc2V0IjotNCwiYW5nbGUiOi05MH1dLFswLDAsImYiLDAseyJvZmZzZXQiOi00LCJhbmdsZSI6OTB9XSxbMCwwLCJlIiwwLHsib2Zmc2V0IjotNCwiYW5nbGUiOjE4MH1dXQ==
\begin{tikzcd}
	\star
	\arrow["fg", shift left=2, from=1-1, to=1-1, loop, in=55, out=125, distance=10mm]
	\arrow["g", shift left=2, from=1-1, to=1-1, loop, in=145, out=215, distance=10mm]
	\arrow["f", shift left=2, from=1-1, to=1-1, loop, in=325, out=35, distance=10mm]
	\arrow["e", shift left=2, from=1-1, to=1-1, loop, in=235, out=305, distance=10mm]
\end{tikzcd} &
% https://q.uiver.app/#q=WzAsMSxbMCwwLCJcXHNxdWFyZSJdLFswLDAsIjAiLDAseyJvZmZzZXQiOi0yfV0sWzAsMCwiMSIsMCx7Im9mZnNldCI6LTIsImFuZ2xlIjo5MH1dLFswLDAsIjIiLDAseyJvZmZzZXQiOi0yLCJyYWRpdXMiOjUsImFuZ2xlIjo5MH1dLFswLDAsIjMiLDAseyJvZmZzZXQiOi0yLCJyYWRpdXMiOjUsImFuZ2xlIjo5MH1dXQ==
\begin{tikzcd}
	\square
	\arrow["0", shift left=2, from=1-1, to=1-1, loop, in=55, out=125, distance=10mm]
	\arrow["1", shift left=2, from=1-1, to=1-1, loop, in=325, out=35, distance=10mm]
	\arrow["2", shift left=2, from=1-1, to=1-1, loop, in=320, out=40, distance=20mm]
	\arrow["3", shift left=2, from=1-1, to=1-1, loop, in=320, out=40, distance=30mm]
\end{tikzcd}
\end{tabular}    
\end{center}
\end{example}

\begin{definition} [Category of Opens]
    Fix $X$ as a topological space. The category of opens
    \begin{itemize}
        \item $\obj (X) = $\{open subsets of $X$\}
        \item $\mor (X) = \mor (u,v) = \left\{ \begin{array}{lr} u \to v, & u\le v \\ \emptyset, & \text{otherwise} \end{array} \right.$
    \end{itemize}
\end{definition}

\begin{definition} [Category of Matrices]
    We can also define a category of matrices
    \begin{itemize}
        \item $\obj (\mat_k) = \mathbb{N} \cup \{0\}$
        \item $\mor (\mat_k) = \mat (m,n) = \left\{\text{n x m matrices with entries in }k \right\}$
    \end{itemize}
\end{definition}

Now, we ask the question, how do we compare two categories? How do we know when they are the same? Consider groups, which are equal if there is an isomorphism, a bijective homomorphism, where bijectivity is injective and surjective. 

\begin{definition}
    A function is injective if $f(x) = f(y) \in H \implies x = y \in G.$
\end{definition}

\begin{definition}
    A function is surjective if $\forall h\in H, \exists g \in G$ s.t. $\phi (g) = h$
\end{definition}

However, both these definitions require information about the objects, not the morphism itself. 